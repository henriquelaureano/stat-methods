\documentclass[12pt, oldfontcommands]{article}\usepackage[]{graphicx}\usepackage[]{color}
%% maxwidth is the original width if it is less than linewidth
%% otherwise use linewidth (to make sure the graphics do not exceed the margin)
\makeatletter
\def\maxwidth{ %
  \ifdim\Gin@nat@width>\linewidth
    \linewidth
  \else
    \Gin@nat@width
  \fi
}
\makeatother

\definecolor{fgcolor}{rgb}{0.345, 0.345, 0.345}
\newcommand{\hlnum}[1]{\textcolor[rgb]{0.686,0.059,0.569}{#1}}%
\newcommand{\hlstr}[1]{\textcolor[rgb]{0.192,0.494,0.8}{#1}}%
\newcommand{\hlcom}[1]{\textcolor[rgb]{0.678,0.584,0.686}{\textit{#1}}}%
\newcommand{\hlopt}[1]{\textcolor[rgb]{0,0,0}{#1}}%
\newcommand{\hlstd}[1]{\textcolor[rgb]{0.345,0.345,0.345}{#1}}%
\newcommand{\hlkwa}[1]{\textcolor[rgb]{0.161,0.373,0.58}{\textbf{#1}}}%
\newcommand{\hlkwb}[1]{\textcolor[rgb]{0.69,0.353,0.396}{#1}}%
\newcommand{\hlkwc}[1]{\textcolor[rgb]{0.333,0.667,0.333}{#1}}%
\newcommand{\hlkwd}[1]{\textcolor[rgb]{0.737,0.353,0.396}{\textbf{#1}}}%
\let\hlipl\hlkwb

\usepackage{framed}
\makeatletter
\newenvironment{kframe}{%
 \def\at@end@of@kframe{}%
 \ifinner\ifhmode%
  \def\at@end@of@kframe{\end{minipage}}%
  \begin{minipage}{\columnwidth}%
 \fi\fi%
 \def\FrameCommand##1{\hskip\@totalleftmargin \hskip-\fboxsep
 \colorbox{shadecolor}{##1}\hskip-\fboxsep
     % There is no \\@totalrightmargin, so:
     \hskip-\linewidth \hskip-\@totalleftmargin \hskip\columnwidth}%
 \MakeFramed {\advance\hsize-\width
   \@totalleftmargin\z@ \linewidth\hsize
   \@setminipage}}%
 {\par\unskip\endMakeFramed%
 \at@end@of@kframe}
\makeatother

\definecolor{shadecolor}{rgb}{.97, .97, .97}
\definecolor{messagecolor}{rgb}{0, 0, 0}
\definecolor{warningcolor}{rgb}{1, 0, 1}
\definecolor{errorcolor}{rgb}{1, 0, 0}
\newenvironment{knitrout}{}{} % an empty environment to be redefined in TeX

\usepackage{alltt}
\usepackage[brazilian, brazil]{babel}
\usepackage[utf8]{inputenc}
\usepackage[T1]{fontenc}
\usepackage{amsmath}
\usepackage{amssymb}
\usepackage{bm}
\usepackage{graphicx}
\usepackage[top = 2cm, left = 2cm, right = 2cm, bottom = 2cm]{geometry}
\usepackage{indentfirst}
\usepackage{float}
\usepackage{multicol}
\usepackage[normalem]{ulem}
\usepackage{breqn}
\usepackage{amsfonts}
\usepackage{amsthm}
\usepackage{enumitem}
\usepackage{booktabs}
\usepackage{threeparttable}
\setlength\parindent{0pt}
\newcommand{\horrule}[1]{\rule{\linewidth}{#1}}

\title{  
 \normalfont \normalsize 
 \textsc{MI404 - Métodos Estatísticos} \\
 Mariana Rodrigues Motta \\
 Departamento de Estatística, Universidade de Campinas (UNICAMP) \\[25pt]
 \horrule{.5pt} \\ [.4cm]
 \LARGE EXERCÍCIOS \\
  2
 \horrule{2pt} \\[ .5cm]}
 
\author{Henrique Aparecido Laureano}
\date{\normalsize Maio de 2017}
\IfFileExists{upquote.sty}{\usepackage{upquote}}{}
\begin{document}

\maketitle

\vspace{\fill}

\tableofcontents

\horrule{1pt} \\

\newpage

\section*{Exercício 1} \addcontentsline{toc}{section}{Exercício 1}

\horrule{1pt} \\

\textbf{Considere o seguinte modelo linear}

\begin{equation}
 \mathbf{y} = \mathbf{X} \bm{\beta} + \bm{\epsilon},
 \label{eq:exer1_1}
\end{equation}

\textbf{em que \(\bm{\epsilon} \sim N(\mathbf{0},
                 \sigma_{e}^{2}\mathbf{I}_{n \times n})\). Seja 
        \(\mathbf{X}\) uma matriz de desenho de dimensão \(n \times p\),
        \(\bm{\beta}\) um vetor de parâmetros de dimensão
        \(p \times 1\).}

\subsection*{(a)} \addcontentsline{toc}{subsection}{(a)}

\textbf{Encontre o EMV de \(\bm{\beta}\) e \(\sigma_{e}^{2}\) e usando a
        verossimilhança baseada na distribuição multivariada de   
        \(\mathbf{y}\).} \\

\textbf{Nota. Seja \(\mathbf{V}\) um vetor aleatório de dimensão
        \(n \times 1\), tal que
        \(\mathbf{V} \sim N(\bm{\mu}, \bm{\Sigma})\), com
        \(\bm{\Sigma}\) positiva definida. Então}

\begin{equation}
 f(\mathbf{v}, \bm{\mu}, \bm{\Sigma}) =
 (2 \pi)^{-n/2} |\bm{\Sigma}|^{-1/2} {\rm exp}\Big\{-\frac{1}{2}
 (\mathbf{v} - \bm{\mu})^{'} \bm{\Sigma}^{-1} (\mathbf{v} - \bm{\mu})
 \Big\}.
 \label{eq:exer1_2}
\end{equation}

\underline{Solução:} \\

Aqui, \(\bm{\theta} = (\bm{\beta}, \sigma_{e}^{2})\),
\(\bm{\mu} = \mathbf{X} \bm{\beta}\) e
\(\bm{\Sigma} = \sigma_{e}^{2}\mathbf{I}_{n \times n}\). \\

A função de verossimilhança \(L(\bm{\theta}; \mathbf{y})\) é dada por:

\[ L(\bm{\theta}; \mathbf{y}) = (2 \pi)^{-n/2}
   |\sigma_{e}^{2}\mathbf{I}_{n \times n}|^{-1/2} {\rm exp}\Big\{
   -\frac{1}{2}
   (\mathbf{y} - \mathbf{X} \bm{\beta})^{'}
   (\sigma_{e}^{2}\mathbf{I}_{n \times n})^{-1}
   (\mathbf{y} - \mathbf{X} \bm{\beta})\Big\}. \]

Sendo \(\sigma_{e}^{2}\mathbf{I}_{n \times n} = {\sigma_{e}^{2}}^{n}\),

\[ L(\bm{\theta}; \mathbf{y}) \propto \frac{1}{|\sigma_{e}^{2}|^{n/2}}
   {\rm exp}\Big\{-\frac{1}{2\sigma_{e}^{2}}
   (\mathbf{y} - \mathbf{X} \bm{\beta})^{'}
   (\mathbf{y} - \mathbf{X} \bm{\beta})\Big\}. \]

A função de log-verossimilhança é expressa por:

\begin{align*}
 {\rm log}(L(\bm{\theta}; \mathbf{y})) & \propto -\frac{n}{2}
 {\rm log}(\sigma_{e}^{2}) -\frac{1}{2\sigma_{e}^{2}}
 (\mathbf{y} - \mathbf{X} \bm{\beta})^{'}
 (\mathbf{y} - \mathbf{X} \bm{\beta}) \\ & \propto -\frac{n}{2}
 {\rm log}(\sigma_{e}^{2}) -\frac{1}{2\sigma_{e}^{2}}
 (\mathbf{y}^{'}\mathbf{y} - 2\mathbf{y}^{'}\mathbf{X}\bm{\beta} +
 (\mathbf{X}\bm{\beta})^{2}).
\end{align*}

\(\boxed{\text{EMV de } \bm{\beta}}\) :

\[ \frac{\partial
   {\rm log}(L(\bm{\theta}; \mathbf{y}))}{\partial \bm{\beta}} = 0 \]

\begin{align*}
 \frac{\partial
 {\rm log}(L(\bm{\theta}; \mathbf{y}))}{\partial \bm{\beta}} & =
 -\frac{1}{2\sigma_{e}^{2}} (-2\mathbf{y}^{'}\mathbf{X} +
 2\mathbf{X}^{'}\mathbf{X}\bm{\beta}) \\ & = \frac{1}{\sigma_{e}^{2}}
 (\mathbf{X}^{'}\mathbf{y} - \mathbf{X}^{'}\mathbf{X}\bm{\beta}).
\end{align*}

\[ \frac{\partial
   {\rm log}(L(\bm{\theta}; \mathbf{y}))}{\partial \bm{\beta}} =
   \frac{1}{\sigma_{e}^{2}}
   (\mathbf{X}^{'}\mathbf{y} - \mathbf{X}^{'}\mathbf{X}\bm{\beta}) = 0.
\]

\begin{align*}
 \frac{1}{\sigma_{e}^{2}}
 (\mathbf{X}^{'}\mathbf{y} - \mathbf{X}^{'}\mathbf{X}\bm{\beta}) & = 0 
 \\ \mathbf{X}^{'}\mathbf{y} - \mathbf{X}^{'}\mathbf{X}\bm{\beta} & = 0
 \\ \mathbf{X}^{'}\mathbf{y} & = \mathbf{X}^{'}\mathbf{X}\bm{\beta}
\end{align*}

\[\boxed{\hat{\bm{\beta}} =
         (\mathbf{X}^{'}\mathbf{X})^{-1}\mathbf{X}^{'}\mathbf{y}.} \]

\(\boxed{\text{EMV de } \sigma_{e}^{2}}\) :

\[ \frac{\partial
   {\rm log}(L(\bm{\theta}; \mathbf{y}))}{\partial \sigma_{e}^{2}} = 0
\]

\begin{align*}
 \frac{\partial
 {\rm log}(L(\bm{\theta}; \mathbf{y}))}{\partial \sigma_{e}^{2}} & =
 -\frac{n}{2\sigma_{e}^{2}} + \frac{1}{2(\sigma_{e}^{2})^{2}}
 (\mathbf{y}^{'}\mathbf{y} - 2\mathbf{y}^{'}\mathbf{X}\bm{\beta} +
 (\mathbf{X}\bm{\beta})^{2}) \\ & = \frac{-n\sigma_{e}^{2} +
 (\mathbf{y}^{'}\mathbf{y} - 2\mathbf{y}^{'}\mathbf{X}\bm{\beta} +
 (\mathbf{X}\bm{\beta})^{2})}{2(\sigma_{e}^{2})^{2}}.
\end{align*}

\[ \frac{\partial
   {\rm log}(L(\bm{\theta}; \mathbf{y}))}{\partial \sigma_{e}^{2}} =
   \frac{-n\sigma_{e}^{2} + (\mathbf{y}^{'}\mathbf{y} -
   2\mathbf{y}^{'}\mathbf{X}\bm{\beta} +
   (\mathbf{X}\bm{\beta})^{2})}{2(\sigma_{e}^{2})^{2}} = 0. \]

\begin{align*}
 \frac{-n\sigma_{e}^{2} + (\mathbf{y}^{'}\mathbf{y} -
 2\mathbf{y}^{'}\mathbf{X}\bm{\beta} +
 (\mathbf{X}\bm{\beta})^{2})}{2(\sigma_{e}^{2})^{2}} & = 0 \\
 -n\sigma_{e}^{2} + (\mathbf{y}^{'}\mathbf{y} -
 2\mathbf{y}^{'}\mathbf{X}\bm{\beta} + (\mathbf{X}\bm{\beta})^{2}) & = 0
 \\ \mathbf{y}^{'}\mathbf{y} - 2\mathbf{y}^{'}\mathbf{X}\bm{\beta} +
 (\mathbf{X}\bm{\beta})^{2} & = n\sigma_{e}^{2} \\
 (\mathbf{y} - \mathbf{X} \bm{\beta})^{'}
 (\mathbf{y} - \mathbf{X} \bm{\beta}) & = n\sigma_{e}^{2}
\end{align*}

\[\boxed{\hat{\sigma}_{e}^{2} = \frac{
         (\mathbf{y} - \mathbf{X} \bm{\beta})^{'}
         (\mathbf{y} - \mathbf{X} \bm{\beta})}{n}.} \]

\subsection*{(b)} \addcontentsline{toc}{subsection}{(b)}

\textbf{Encontre a distribuição do EMV \(\hat{\bm{\beta}}\).} \\

\underline{Solução:}

\[ \hat{\bm{\beta}} \underset{\text{aprox.}}{\sim}
    N(E[\hat{\bm{\beta}}], Var[\hat{\bm{\beta}}]) \]

\begin{multicols}{2}
\begin{align*}
 E[\hat{\bm{\beta}}] & =
 E[(\mathbf{X}^{'}\mathbf{X})^{-1}\mathbf{X}^{'}\mathbf{y}] \\ & =
 (\mathbf{X}^{'}\mathbf{X})^{-1}\mathbf{X}^{'}E[\mathbf{y}] \\ & =
 (\mathbf{X}^{'}\mathbf{X})^{-1}\mathbf{X}^{'}
 E[\mathbf{X} \bm{\beta} + \bm{\epsilon}] \\ & =
 (\mathbf{X}^{'}\mathbf{X})^{-1}\mathbf{X}^{'}\mathbf{X}E[\bm{\beta}] +
 \mathbf{0} \\ & = (\mathbf{X}^{'}\mathbf{X})^{-1}
 \mathbf{X}^{'}\mathbf{X}\bm{\beta} \\ & = \bm{\beta}.
 \\ \\ \\ \\ \\ \\ \\ \\ \\ \\ \\
\end{align*}

\columnbreak

\[ Var[\hat{\bm{\beta}}] = \frac{1}{\mathbf{I}(\bm{\beta})}, \qquad
   \mathbf{I}(\bm{\beta}) = E\bigg[-\frac{\partial^2
   {\rm log}(L(\bm{\theta};\mathbf{y}))}{\partial \bm{\beta} \partial
   \bm{\beta}^{'}}\bigg]. \]

Sendo que \(\mathbf{I}(\bm{\beta})\) é a matriz de informação esperada.

\begin{align*}
 \frac{\partial^2 {\rm log}(L(\bm{\theta};\mathbf{y}))}{\partial
 \bm{\beta} \partial \bm{\beta}^{'}} & = -\frac{\mathbf{X}^{'}
 \mathbf{X}}{\sigma_{e}^{2}} \\ \mathbf{I}(\bm{\beta}) & =
 E\bigg[\frac{\mathbf{X}^{'}\mathbf{X}}{\sigma_{e}^{2}}\bigg] \\
 \mathbf{I}(\bm{\beta}) & = \frac{\mathbf{X}^{'}\mathbf{X}}{E\Big[
 \sigma_{e}^{2}\Big]} \\ \mathbf{I}(\bm{\beta}) & =
 \frac{\mathbf{X}^{'}\mathbf{X}}{\sigma_{e}^{2}}.
\end{align*}

\begin{align*}
 Var[\hat{\bm{\beta}}] & =
 \frac{1}{\mathbf{X}^{'}\mathbf{X}/\sigma_{e}^{2}} \\ & =
 (\mathbf{X}^{'}\mathbf{X})^{-1}\sigma_{e}^{2}.
\end{align*}
\end{multicols}

\[ \boxed{\hat{\bm{\beta}} \underset{\text{aprox.}}{\sim}
          N\Big(\bm{\beta},
                (\mathbf{X}^{'}\mathbf{X})^{-1}\sigma_{e}^{2}\Big).} \]

\subsection*{(c)} \addcontentsline{toc}{subsection}{(c)}

\textbf{Encontre a distribuição do EMV \(\hat{\sigma}_{e}^{2}\).} \\

\underline{Solução:}

\[ \hat{\sigma}_{e}^{2} \underset{\text{aprox.}}{\sim}
    N(E[\hat{\sigma}_{e}^{2}], Var[\hat{\sigma}_{e}^{2}]) \]

\begin{align*}
 E[\hat{\sigma}_{e}^{2}] = E\bigg[\frac{
 (\mathbf{y} - \mathbf{X} \hat{\bm{\beta}})^{'}
 (\mathbf{y} - \mathbf{X} \hat{\bm{\beta}})}{n}\bigg] & = \frac{1}{n}
 E\Big[(\mathbf{y} - \mathbf{X} \hat{\bm{\beta}})^{'}
       (\mathbf{y} - \mathbf{X} \hat{\bm{\beta}})\Big] \\ & =
 \frac{1}{n} E\Bigg[\sum_{i = 1}^{n}
 (y_{i} - \bm{x}_{i}^{'}\hat{\bm{\beta}})^{2}\Bigg] \\ & = \frac{1}{n}
 E\Bigg[\frac{\sigma^{2}}{\sigma^{2}} \sum_{i = 1}^{n}
        (y_{i} - \bm{x}_{i}^{'}\hat{\bm{\beta}})^{2}\Bigg] \\ & =
 \frac{1}{n} \sigma^{2} E\left[\sum_{i = 1}^{n}
 \Bigg[\frac{(y_{i} - \bm{x}_{i}^{'}\hat{\bm{\beta}})}{\sigma}\Bigg]^{2}
 \right] = \frac{1}{n} \sigma^{2} E\Big[\chi_{n - p}^{2}\Big].
\end{align*}

Já que

\[ \sum_{i = 1}^{n} \Bigg[
   \frac{(y_{i} -\bm{x}_{i}^{'}\hat{\bm{\beta}})}{\sigma}\Bigg]^{2} \sim
   \chi_{n - p}^{2}, \quad {\rm com} \quad E[\cdot] = n - p \quad
   {\rm e} \quad Var[\cdot] = 2 (n - p).\]

Assim,

\begin{multicols}{2}
\begin{align*}
 E[\hat{\sigma}_{e}^{2}] & = \frac{1}{n} \sigma^{2} (n - p) \\
                         & = \frac{n - p}{n} \sigma^{2}.
\\ \\ \\ \\ \\ \\ \\ \\
\end{align*}
\begin{align*}
 Var[\hat{\sigma}_{e}^{2}] & =
 Var\bigg[\frac{(\mathbf{y} - \mathbf{X} \hat{\bm{\beta}})^{'}
                (\mathbf{y} - \mathbf{X} \hat{\bm{\beta}})}{n}\bigg] \\
 & = \frac{1}{n^{2}} Var\bigg[
     \sum_{i = 1}^{n} (y_{i} - \bm{x}_{i}^{'}\hat{\bm{\beta}})^{2}\bigg]
 \\ & = \frac{1}{n^{2}} \sigma^{4} Var\left[\sum_{i = 1}^{n}
 \Bigg[\frac{(y_{i} - \bm{x}_{i}^{'}\hat{\bm{\beta}})}{\sigma}\Bigg]^{2}
 \right] \\ & = \frac{1}{n^{2}} \sigma^{4} 2 (n - p) \\
 & = \frac{n - p}{n^{2}} 2 \sigma^{4}.
\end{align*}
\end{multicols}

\[ \boxed{\hat{\sigma}_{e}^{2} \underset{\text{aprox.}}{\sim}
          N\bigg(\frac{n - p}{n} \sigma^{2},
                \frac{n - p}{n^{2}} 2 \sigma^{4}\bigg).} \]

\subsection*{(d)} \addcontentsline{toc}{subsection}{(d)}

\textbf{Considere o modelo em (\ref{eq:exer1_1}) com
        \(\bm{\beta} = \beta_{0}\) e estimador restrito de
        \(\sigma^{2} (\hat{\sigma}_{r}^{2})\) visto em aula. Encontre
        \(Var[\hat{\sigma}_{r}^{2}]\) e \(Var[\hat{\sigma}_{e}^{2}]\).
        Na sua opinião, qual deles é melhor? Justifique.} \\

\underline{Solução:}

\[ \mathbf{y} = \beta_{0} + \bm{\epsilon}, \quad
   \bm{\epsilon} \sim
   N(\mathbf{0}, \sigma_{e}^{2}\mathbf{I}_{n \times n}) \]

\begin{multicols}{2}
EMV de \(\sigma_{e}^{2}\):

\[ \hat{\sigma}_{e}^{2} =
   \sum_{i = 1}^{n} \frac{(y_{i} - \hat{\beta_{0}})^{2}}{n} \]

\begin{align*}
 Var[\hat{\sigma}_{e}^{2}] & =
 Var[\sum_{i = 1}^{n} \frac{(y_{i} - \hat{\beta_{0}})^{2}}{n}] \\ & =
 \frac{1}{n^{2}} \sigma^{4} 2 (n - 1)
\end{align*}

\[ \boxed{Var[\hat{\sigma}_{e}^{2}] =
          \frac{2 (n - 1)}{n^{2}} \sigma^{4}.} \]

EMV restrito, \(\sigma_{r}^{2}\):

\[ \hat{\sigma}_{r}^{2} =
   \sum_{i = 1}^{n} \frac{(y_{i} - \hat{\beta_{0}})^{2}}{n - 1} \]

\begin{align*}
 Var[\hat{\sigma}_{r}^{2}] & =
 Var[\sum_{i = 1}^{n} \frac{(y_{i} - \hat{\beta_{0}})^{2}}{n - 1}] \\
 & = \frac{1}{(n - 1)^{2}} \sigma^{4} 2 (n - 1)
\end{align*}

\[ \boxed{Var[\hat{\sigma}_{r}^{2}] = \frac{2}{n - 1} \sigma^{4}.} \]
\end{multicols}

\begin{align*}
 \frac{Var[\hat{\sigma}_{e}^{2}]}{Var[\hat{\sigma}_{r}^{2}]} & =
 \frac{2 (n - 1)}{n^{2}} \sigma^{4} \cdot \frac{n - 1}{2 \sigma^{4}} \\
 & = \frac{(n - 1)^{2}}{n^{2}} \\ & = \frac{n^{2} - 2n + 1}{n^{2}} \\
 & = 1 - \frac{2}{n} + \frac{1}{n^{2}} \\ & = \leq 1.
\end{align*}

Portanto,
\(\boxed{Var[\hat{\sigma}_{r}^{2}] > Var[\hat{\sigma}_{e}^{2}]}\). \\

\(\hat{\sigma}_{e}^{2}\),
\(E[\hat{\sigma}_{e}^{2}] = \frac{n - 1}{n} \sigma_{e}^{2}\), é um
estimador viciado (corrigível), mas com menor variância que
\(\hat{\sigma}_{r}^{2}\), \(E[\hat{\sigma}_{r}^{2}] = \sigma_{e}^{2}\).
\\

Logo, temos que
\(\hat{\sigma}_{e}^{2}\) é melhor que \(\hat{\sigma}_{r}^{2}\).

\section*{Exercício 2} \addcontentsline{toc}{section}{Exercício 2}

\horrule{1pt} \\

\textbf{Considere o modelo dado em (\ref{eq:exer1_1}). A partir da
        perspectiva Bayesiana, considere as distribuições à priori de
        \(\bm{\beta}\) e \(\sigma_{e}^{2}\) dados por
        \(p(\bm{\beta}) \propto 1\) e
        \(p(\sigma_{e}^{2}) \propto (\sigma_{e}^{2})^{-1}\),
        respectivamente.}

\subsection*{(a)} \addcontentsline{toc}{subsection}{(a)}

\textbf{Seja \(\bm{\theta} = (\bm{\beta}^{'}, \sigma_{e}^{2})\).
        Encontre a distribuição a posteriori de \(\bm{\theta}\).} \\

\underline{Solução:} \\

A distribuição a posteriori de \(\bm{\theta}\),
\(\pi(\bm{\theta} | \mathbf{y})\), pelo teorema de Bayes é dada por:

\[ \pi(\bm{\theta} | \mathbf{y}) \propto
   L(\bm{\theta}; \mathbf{y}) \pi(\bm{\theta}). \]

Assumindo independência entre \(\bm{\theta}\) e \(\sigma_{e}^{2}\), a
distribuição a priori de \(\pi(\bm{\theta})\) é dada por:

\[ \pi(\bm{\theta}) = \pi(\bm{\beta}, \sigma_{e}^{2}) =
   \pi(\bm{\beta}) \pi(\sigma_{e}^{2}) \propto
   1 \cdot \frac{1}{\sigma_{e}^{2}} = \frac{1}{\sigma_{e}^{2}}. \]

Assim,

\begin{align*}
 \pi(\bm{\theta} | \mathbf{y}) & \propto
 \frac{1}{(\sigma_{e}^{2})^{n/2}}
 {\rm exp}\Big\{-\frac{1}{2\sigma_{e}^{2}}
 (\mathbf{y} - \mathbf{X} \bm{\beta})^{'}
 (\mathbf{y} - \mathbf{X} \bm{\beta})\Big\} \cdot
 \frac{1}{\sigma_{e}^{2}} \\ & \propto
 \frac{1}{(\sigma_{e}^{2})^{n/2 + 1}}
 {\rm exp}\Big\{-\frac{1}{2\sigma_{e}^{2}}
 (\mathbf{y} - \mathbf{X} \bm{\beta})^{'}
 (\mathbf{y} - \mathbf{X} \bm{\beta})\Big\}.
\end{align*}

Sabemos que \(\hat{\bm{\beta}} =
              (\mathbf{X}^{'}\mathbf{X})^{-1}\mathbf{X}^{'}\mathbf{y}\)
e \(\hat{\mathbf{y}} = \mathbf{X}\hat{\bm{\beta}}\), desta forma:

\begin{align*}
 (\mathbf{y} - \mathbf{X} \bm{\beta})^{'}
 (\mathbf{y} - \mathbf{X} \bm{\beta}) & =
 (\mathbf{y} - \mathbf{X} \bm{\beta})^{'}
 (\mathbf{y} - \mathbf{X} \bm{\beta})
 - 2\mathbf{y}^{'}\hat{\mathbf{y}} + 2\mathbf{y}^{'}\hat{\mathbf{y}} \\
 & = \mathbf{y}^{'}\mathbf{y} - 2\mathbf{y}\bm{\beta}^{'}\mathbf{X}^{'}
     + \bm{\beta}^{'}\mathbf{X}^{'}\mathbf{X}\bm{\beta}
     - 2\mathbf{y}^{'}\mathbf{X}\hat{\bm{\beta}}
     + 2\mathbf{y}^{'}\mathbf{X}\hat{\bm{\beta}} \\ & =
 \mathbf{y}^{'}\mathbf{y} - 2\bm{\beta}^{'}\mathbf{X}^{'}
 \mathbf{X}(\mathbf{X}^{'}\mathbf{X})^{-1}\mathbf{X}^{'}\mathbf{y}
 + \bm{\beta}^{'}\mathbf{X}^{'}\mathbf{X}\bm{\beta}
 - 2\mathbf{y}^{'}\mathbf{X}\hat{\bm{\beta}}
 + 2\mathbf{y}^{'}\mathbf{X}
 (\mathbf{X}^{'}\mathbf{X})^{-1}\mathbf{X}^{'}\mathbf{X}\hat{\bm{\beta}}
 \\ & = \mathbf{y}^{'}\mathbf{y}
        - 2\bm{\beta}^{'}\mathbf{X}^{'}\mathbf{X}\hat{\bm{\beta}}
        + \bm{\beta}^{'}\mathbf{X}^{'}\mathbf{X}\bm{\beta}
        - 2\mathbf{y}^{'}\mathbf{X}\hat{\bm{\beta}}
        + 2\hat{\bm{\beta}}^{'}\mathbf{X}^{'}\mathbf{X}\hat{\bm{\beta}}
 \\ & = \mathbf{y}^{'}\mathbf{y}
        - \bm{\beta}^{'}\mathbf{X}^{'}\mathbf{X}\hat{\bm{\beta}}
        - \hat{\bm{\beta}}^{'}\mathbf{X}^{'}\mathbf{X}\bm{\beta}
        + \bm{\beta}^{'}\mathbf{X}^{'}\mathbf{X}\bm{\beta}
        - \mathbf{y}^{'}\mathbf{X}\hat{\bm{\beta}}
        - \hat{\bm{\beta}}^{'}\mathbf{X}^{'}\mathbf{y}
        + \hat{\bm{\beta}}^{'}\mathbf{X}^{'}\mathbf{X}\hat{\bm{\beta}}
        + \hat{\bm{\beta}}^{'}\mathbf{X}^{'}\mathbf{X}\hat{\bm{\beta}}
 \\ & = \mathbf{y}^{'}\mathbf{y}
        - \mathbf{y}^{'}\mathbf{X}\hat{\bm{\beta}}
        - \hat{\bm{\beta}}^{'}\mathbf{X}^{'}\mathbf{y}
        + \hat{\bm{\beta}}^{'}\mathbf{X}^{'}\mathbf{X}\hat{\bm{\beta}}
        + \bm{\beta}^{'}\mathbf{X}^{'}\mathbf{X}\bm{\beta}
        - \bm{\beta}^{'}\mathbf{X}^{'}\mathbf{X}\hat{\bm{\beta}}
        - \hat{\bm{\beta}}^{'}\mathbf{X}^{'}\mathbf{X}\bm{\beta}
        + \hat{\bm{\beta}}^{'}\mathbf{X}^{'}\mathbf{X}\hat{\bm{\beta}}
 \\ & = (\mathbf{y} - \mathbf{X} \hat{\bm{\beta}})^{'}
        (\mathbf{y} - \mathbf{X} \hat{\bm{\beta}})
        + (\bm{\beta} - \hat{\bm{\beta}})^{'}
          \mathbf{X}^{'}\mathbf{X}(\bm{\beta} - \hat{\bm{\beta}}) \\
 & = \frac{n - p}{n - p}
     (\mathbf{y} - \hat{\mathbf{y}})^{'}(\mathbf{y} - \hat{\mathbf{y}})
     + (\bm{\beta} - \hat{\bm{\beta}})^{'}
       \mathbf{X}^{'}\mathbf{X}(\bm{\beta} - \hat{\bm{\beta}}) \\
 & = \nu \mathbf{S}^{2} + (\bm{\beta} - \hat{\bm{\beta}})^{'}
     \mathbf{X}^{'}\mathbf{X}(\bm{\beta} - \hat{\bm{\beta}}),
\end{align*}

com

\[ \mathbf{S}^{2} =
   \frac{
    (\mathbf{y}-\hat{\mathbf{y}})^{'}(\mathbf{y}-\hat{\mathbf{y}})}{
    \nu}, \quad \nu = n - p. \]

Portanto,

\[ \boxed{\pi(\bm{\theta} | \mathbf{y}) \propto
    \frac{1}{(\sigma_{e}^{2})^{n/2 + 1}}
    {\rm exp}\bigg\{-\frac{1}{2\sigma_{e}^{2}}
    \Big[\nu \mathbf{S}^{2} + (\bm{\beta} - \hat{\bm{\beta}})^{'}
         \mathbf{X}^{'}\mathbf{X}(\bm{\beta} - \hat{\bm{\beta}})
    \Big]\bigg\}}. \]

\subsection*{(b)} \addcontentsline{toc}{subsection}{(b)}

\textbf{Usando o item (a), encontre a distribuição a posteriori de
        \(\sigma_{e}^{2}\).} \\

\underline{Solução:}

\vspace{-.18cm}

\begin{align}
\begin{split}
 \pi(\sigma_{e}^{2}|\mathbf{y}) =
 \int_{-\infty}^{\infty} \pi(\bm{\beta}, \sigma_{e}^{2}|\mathbf{y})
                         {\rm d}\bm{\beta} & =
 \int_{-\infty}^{\infty} \frac{1}{(\sigma_{e}^{2})^{n/2 + 1}}
  {\rm exp}\bigg\{-\frac{1}{2\sigma_{e}^{2}}
  \Big[\nu \mathbf{S}^{2} + (\bm{\beta} - \hat{\bm{\beta}})^{'}
  \mathbf{X}^{'}\mathbf{X}(\bm{\beta} - \hat{\bm{\beta}}) \Big]\bigg\}
 {\rm d}\bm{\beta} \\ & = \frac{1}{(\sigma_{e}^{2})^{n/2 + 1}}
 {\rm exp}\bigg\{-\frac{\nu \mathbf{S}^{2}}{2\sigma_{e}^{2}}\bigg\}
 \int_{-\infty}^{\infty} {\rm exp}\bigg\{-\frac{1}{2}
  (\bm{\beta} - \hat{\bm{\beta}})^{'}
  \frac{\mathbf{X}^{'}\mathbf{X}}{\sigma_{e}^{2}}
  (\bm{\beta} - \hat{\bm{\beta}})\bigg\}{\rm d}\bm{\beta} \\ & =
  \frac{1}{(\sigma_{e}^{2})^{n/2 + 1}}
 {\rm exp}\bigg\{-\frac{\nu \mathbf{S}^{2}}{2\sigma_{e}^{2}}\bigg\}
 (\sqrt{2\pi})^{p}
 \bigg|\frac{(\mathbf{X}^{'}\mathbf{X})^{-1}}{\sigma_{e}^{2}}\bigg|^{1/2}
 \\ & \propto \frac{1}{(\sigma_{e}^{2})^{n/2 + 1}}
 {\rm exp}\bigg\{-\frac{\nu \mathbf{S}^{2}}{2\sigma_{e}^{2}}\bigg\}
 (\sqrt{2\pi})^{p} \frac{1}{((\sigma_{e}^{2})^{2p})^{-1/2}} \\ & =
 \frac{1}{(\sigma_{e}^{2})^{n/2 + 1 - p}}
 {\rm exp}\bigg\{-\frac{\nu \mathbf{S}^{2}}{2\sigma_{e}^{2}}\bigg\} \\
 & = \frac{1}{(\sigma_{e}^{2})^{n/2 + 1 - p/2}}
 {\rm exp}\bigg\{-\frac{\nu \mathbf{S}^{2}}{2\sigma_{e}^{2}}\bigg\}
 \label{eq:2.b} \\ & = \frac{1}{(\sigma_{e}^{2})^{(n - p)/2 + 1}}
 {\rm exp}\bigg\{-\frac{\nu \mathbf{S}^{2}}{2\sigma_{e}^{2}}\bigg\} =
 \frac{1}{(\sigma_{e}^{2})^{\nu/2 + 1}}
 {\rm exp}\bigg\{-\frac{\nu \mathbf{S}^{2}}{2\sigma_{e}^{2}}\bigg\}.
\end{split}
\end{align}

Na equação (\ref{eq:2.b}) temos o núcleo de uma distribuição gama
inversa, assim, a distribuição marginal a posteriori de
\(\sigma_{e}^{2}\) é dada por:

\[ \pi(\sigma_{e}^{2}|\mathbf{y}) =
   \frac{\nu \mathbf{S}^{2}}{2 \Gamma(\nu/2)}
   \Big(\frac{1}{\sigma_{e}^{2}}\Big)^{\nu/2 + 1}
   {\rm exp}\bigg\{-\frac{\nu \mathbf{S}^{2}}{2\sigma_{e}^{2}}\bigg\}, \]

ou seja,

\[ \pi(\sigma_{e}^{2}|\mathbf{y}) =
   GInv\bigg(\frac{\nu}{2}, \frac{\nu \mathbf{S}^{2}}{2}\bigg). \]

\subsection*{(c)} \addcontentsline{toc}{subsection}{(c)}

\textbf{Usando o item (a), encontre a distribuição a posteriori de
        \(\bm{\beta}\).} \\
        
\underline{Solução:}

\[ \pi(\bm{\beta}|\mathbf{y}) =
   \int_{0}^{\infty} \pi(\bm{\beta}, \sigma_{e}^{2}|\mathbf{y})
   {\rm d}\sigma_{e}^{2} = \int_{0}^{\infty}
   \frac{1}{(\sigma_{e}^{2})^{n/2 + 1}}
   {\rm exp}\bigg\{-\frac{\nu \mathbf{S}^{2} +
   (\bm{\beta} - \hat{\bm{\beta}})^{'}\mathbf{X}^{'}\mathbf{X}(\bm{\beta}
   - \hat{\bm{\beta}})}{2\sigma_{e}^{2}} \bigg\}{\rm d}\sigma_{e}^{2}. \]

Fazendo \(a = (\nu \mathbf{S}^{2} +
(\bm{\beta} - \hat{\bm{\beta}})^{'}\mathbf{X}^{'}\mathbf{X}(\bm{\beta} -
\hat{\bm{\beta}}))/2\), temos:

\begin{align}
\begin{split}
 \pi(\bm{\beta}|\mathbf{y}) & = \int_{0}^{\infty}
 (\sigma_{e}^{2})^{-(n/2 + 1)}
 {\rm exp}\bigg\{-\frac{a}{\sigma_{e}^{2}}\bigg\}{\rm d}\sigma_{e}^{2} \\
 & = \bigg(\frac{a}{2}\bigg)^{-n/2}\Gamma\bigg(\frac{n}{2}\bigg) \\
 & = \bigg[\frac{\nu \mathbf{S}^{2} + (\bm{\beta} - \hat{\bm{\beta}})^{'}
     \mathbf{X}^{'}\mathbf{X}(\bm{\beta} - \hat{\bm{\beta}})}{2}
     \bigg]^{n/2} \Gamma\bigg(\frac{n}{2}\bigg) \\ & = \Gamma
 \bigg(\frac{\nu + p}{2}\bigg) 2^{n/2} \bigg\{\nu\mathbf{S}^{2}\bigg[
 1 + \frac{(\bm{\beta} - \hat{\bm{\beta}})^{'}
 \mathbf{X}^{'}\mathbf{X}(\bm{\beta} - \hat{\bm{\beta}})}{
 \nu\mathbf{S}^{2}}\bigg]\bigg\}^{-(1/2)(\nu + p)} \\ & = \Gamma
 \bigg(\frac{\nu + p}{2}\bigg) {\nu\mathbf{S}^{2}}^{-(1/2)(\nu + p)}
 \bigg[1 + \frac{(\bm{\beta} - \hat{\bm{\beta}})^{'}
 \mathbf{X}^{'}\mathbf{X}(\bm{\beta} - \hat{\bm{\beta}})}{
 \nu\mathbf{S}^{2}}\bigg]^{-(1/2)(\nu + p)} \label{eq:2.c} \\ & \propto
 \Gamma\bigg(\frac{1}{2}(\nu + p)\bigg) \nu^{-p/2}\mathbf{S}^{-p}
 \bigg[1 + \frac{(\bm{\beta} - \hat{\bm{\beta}})^{'}
 \mathbf{X}^{'}\mathbf{X}(\bm{\beta} - \hat{\bm{\beta}})}{
 \nu\mathbf{S}^{2}}\bigg]^{-(1/2)(\nu + p)}.
\end{split}
\end{align}

Na equação (\ref{eq:2.c}) temos o núcleo de uma distribuição \textit{t}-
Student multivariada, assim, a distribuição marginal a posteriori de
\(\bm{\beta}\) é dada por:

\[ \pi(\bm{\beta}|\mathbf{Y}) =
   \frac{\Gamma [(1/2)(\nu + p)]|\mathbf{X}^{'}\mathbf{X}|^{1/2}
         \mathbf{S}^{-p}}{\Gamma(1/2)^{-p}\Gamma((1/2)\nu)
                          (\sqrt{\nu})^{p}}
   \bigg[1 + \frac{(\bm{\beta} - \hat{\bm{\beta}})^{'}
   \mathbf{X}^{'}\mathbf{X}(\bm{\beta} - \hat{\bm{\beta}})}{
   \nu\mathbf{S}^{2}}\bigg]^{-(1/2)(\nu + p)}, \]

que se denota por \(t_{p}(\hat{\bm{\beta}},
\mathbf{S}^{2}(\mathbf{X}^{'}\mathbf{X})^{-1}, \nu)\).

\subsection*{(d)} \addcontentsline{toc}{subsection}{(d)}

\textbf{Supondo que fosse necessário apontar um estimador pontual
        Bayesiano para \(\bm{\beta}\), quem você apontaria?} \\

\underline{Solução:} \\

Cada parâmetro \(\beta_{i}, i = 0, 1, ..., p\) tem distribuição,

\[ \pi(\beta_{i}|\mathbf{Y}) = t(\nu, \hat{\beta}_{i}, h_{ii}), \]

\textit{t}-Student univariada com \(\nu\) graus de liberdade, parâmetro
de posição \(\hat{\beta}\) e parâmetro de escala \(h_{ii}\) que é o
elemento \((i, i)\) de \(\mathbf{S}^{2}(\mathbf{X}^{'}\mathbf{X})^{-1}\).
\\

Como a distribuição é uma \textit{t}-Student, um estimador pontual não
viciado bom seria a média desta \textit{t}-Student:

\[ \hat{\bm{\beta}} =
   (\mathbf{X}^{'} \mathbf{X})^{-1} \mathbf{X}^{'} \mathbf{y}. \]

\section*{Exercício 3} \addcontentsline{toc}{section}{Exercício 3}

\horrule{1pt} \\

\textbf{Compare o EMV de \(\bm{\beta}\) e a média da distribuição à
        posteriori de \(\bm{\beta}\) em termos de suas respectivas
        variâncias, indicando qual deles seria mais apropriado.} \\

\underline{Solução:} \\

Como visto anteriormente,

\begin{multicols}{2}
 \[ Var[\hat{\bm{\beta}}_{EMV}] =
    \mathbf{S}^{2} (\mathbf{X}^{'} \mathbf{X})^{-1}. \] \\

\columnbreak

Distribuição à posteriori de \(\bm{\beta}\): \\

Como \(\pi(\beta_{i}|\mathbf{Y}) = t(\nu, \hat{\beta}_{i}, h_{ii})\),

\[ Var[\hat{\bm{\beta}}] =
   \frac{n-p}{n-p-2} \mathbf{S}^{2} (\mathbf{X}^{'} \mathbf{X})^{-1}. \]
\end{multicols}

Assim,

\begin{align*}
 \mathbf{S}^{2} (\mathbf{X}^{'} \mathbf{X})^{-1} & <
 \frac{n - p}{n - p - 2} \mathbf{S}^{2} (\mathbf{X}^{'} \mathbf{X})^{-1}
 \\ 1 & < \frac{n - p}{n - p - 2}.
\end{align*}

\(\hat{\bm{\beta}}_{EMV}\) tem menor variância. Contudo, apesar de sermos
capazes de calcular a esperança e a variância de \(\bm{\beta}\) por EMV,
não conhecemos sua distribuição. Já a partir da perspectiva bayesiana
somos capazes de conhecer sua distribuição a posteriori. Portanto, sob
esse ponto de visto, em situações de amostra pequena ou que seja de
interesse a distribuição de \(\bm{\beta}\), a utilização da perspectiva
bayesiana é mais apropriada.
\hfill \(\blacksquare\)

% \horrule{.5pt}

% \vspace{\fill}

 %\horrule{1pt} % \\

\end{document}
