\documentclass[12pt, oldfontcommands]{article}\usepackage[]{graphicx}\usepackage[]{color}
%% maxwidth is the original width if it is less than linewidth
%% otherwise use linewidth (to make sure the graphics do not exceed the margin)
\makeatletter
\def\maxwidth{ %
  \ifdim\Gin@nat@width>\linewidth
    \linewidth
  \else
    \Gin@nat@width
  \fi
}
\makeatother

\definecolor{fgcolor}{rgb}{0, 0, 0}
\newcommand{\hlnum}[1]{\textcolor[rgb]{0.502,0,0.502}{\textbf{#1}}}%
\newcommand{\hlstr}[1]{\textcolor[rgb]{0.651,0.522,0}{#1}}%
\newcommand{\hlcom}[1]{\textcolor[rgb]{1,0.502,0}{#1}}%
\newcommand{\hlopt}[1]{\textcolor[rgb]{1,0,0.502}{\textbf{#1}}}%
\newcommand{\hlstd}[1]{\textcolor[rgb]{0,0,0}{#1}}%
\newcommand{\hlkwa}[1]{\textcolor[rgb]{0.733,0.475,0.467}{\textbf{#1}}}%
\newcommand{\hlkwb}[1]{\textcolor[rgb]{0.502,0.502,0.753}{\textbf{#1}}}%
\newcommand{\hlkwc}[1]{\textcolor[rgb]{0,0.502,0.753}{#1}}%
\newcommand{\hlkwd}[1]{\textcolor[rgb]{0,0.267,0.4}{#1}}%
\let\hlipl\hlkwb

\usepackage{framed}
\makeatletter
\newenvironment{kframe}{%
 \def\at@end@of@kframe{}%
 \ifinner\ifhmode%
  \def\at@end@of@kframe{\end{minipage}}%
  \begin{minipage}{\columnwidth}%
 \fi\fi%
 \def\FrameCommand##1{\hskip\@totalleftmargin \hskip-\fboxsep
 \colorbox{shadecolor}{##1}\hskip-\fboxsep
     % There is no \\@totalrightmargin, so:
     \hskip-\linewidth \hskip-\@totalleftmargin \hskip\columnwidth}%
 \MakeFramed {\advance\hsize-\width
   \@totalleftmargin\z@ \linewidth\hsize
   \@setminipage}}%
 {\par\unskip\endMakeFramed%
 \at@end@of@kframe}
\makeatother

\definecolor{shadecolor}{rgb}{.97, .97, .97}
\definecolor{messagecolor}{rgb}{0, 0, 0}
\definecolor{warningcolor}{rgb}{1, 0, 1}
\definecolor{errorcolor}{rgb}{1, 0, 0}
\newenvironment{knitrout}{}{} % an empty environment to be redefined in TeX

\usepackage{alltt}
\usepackage[brazilian, brazil]{babel}
\usepackage[utf8]{inputenc}
\usepackage[T1]{fontenc}
\usepackage{amsmath}
\usepackage{graphicx}
\usepackage[top = 2cm, left = 2cm, right = 2cm, bottom = 2cm]{geometry}
\usepackage{indentfirst}
\usepackage{float}
\usepackage{multicol}
\usepackage[normalem]{ulem}
\usepackage{breqn}
\usepackage{amsfonts}
\usepackage{amsthm}
\usepackage{enumitem}
\usepackage{booktabs}
\usepackage{threeparttable}
\setlength\parindent{0pt}
\newcommand{\eqnb}{\begin{equation}}
\newcommand{\eqne}{\end{equation}}
\newcommand{\eqnbs}{\begin{equation*}}
\newcommand{\eqnes}{\end{equation*}}
\newcommand{\horrule}[1]{\rule{\linewidth}{#1}}

\title{  
 \normalfont \normalsize 
 \textsc{MI404 - Métodos Estatísticos} \\
 Mariana Rodrigues Motta \\
 Departamento de Estatística, Universidade de Campinas (UNICAMP) \\ [25pt]
 \horrule{.5pt} \\ [.4cm]
 \LARGE PROVA 1 \\
  - \\
  EXERCÍCIOS
 \horrule{2pt} \\[ .5cm]}
 
\author{Henrique Aparecido Laureano}
\date{\normalsize Abril de 2017}
\IfFileExists{upquote.sty}{\usepackage{upquote}}{}
\begin{document}

\maketitle

\vspace{\fill}

\tableofcontents

\horrule{1pt} \\

\newpage



\section*{Exercício 1} \addcontentsline{toc}{section}{Exercício 1}

\horrule{1pt} \\

\textbf{Considere uma variável aleatória \(X\) = consumo (em gramas) de
        vegetais de \(n\) = 188 indivíduos. A seguir considere o gráfico
        abaixo e responda.}

\vspace{\fill}
\begin{figure}[H]
 \centering
  \includegraphics[width = .65\textwidth]{exer1.png}
 \caption{Histograma do consumo de vegetais (em gramas).}
 \label{fig:exer1}
\end{figure}

\subsection*{(a)} \addcontentsline{toc}{subsection}{(a)}

\textbf{O que você observa sobre o histograma acima?} \\

\quad
Observa-se inicialmente que o consumo de vegetais dos indivíduos sob
estudo varia de O à 400 gramas, com maior concentração entre 50 e 100
gramas. Este intervalo de maior concentração apresenta mais que o dobro de
indivíduos que o segundo intervalo com maior frequência de consumo
(intervalo 100 à 150 gramas). Feita esta observações, podemos afirmar que
o consumo de vegetais destes indivíduos segue uma distribuição unimodal
(com uma única moda).
     
\subsection*{(b)} \addcontentsline{toc}{subsection}{(b)}

\textbf{O que significa a escala do eixo \(y\) ser chamada de
        densidade?} \\
        
\quad
Significa que a escala do histograma apresentado é dada em densidade de
probabilidade, desta modo, o histograma tem uma área total de um.

\subsection*{(c)} \addcontentsline{toc}{subsection}{(c)}

\textbf{Neste exemplo, quem será maior: a média ou a mediana? 
        Justifique. \\
        Considere \(\bar{X}\) = 87.2 e \(S\) = 49.14, em que \(\bar{X}\) é
        a média amostral e \(S\) o desvio padrão amostral.} \\

\quad
Este histograma apresenta assimetria a direita (cauda direita mais longo e
fina), e em histogramas com tal característica a média é maior que a
mediana. Isso é dado pelo fato do conjunto de dados possuir poucas
observações com valores altos que aumentam a média mas que não afetam a
localização do meio exato do conjunto de dados (i.e., a mediana). \\
     
Para ter uma noção da mediana, \(\tilde{x}\), podemos usar a seguinte
inequação:
       
\[ \text{max}(x_{(1)}, \bar{x} - s_{n}) \leq \tilde{x} < \bar{x}, \]

em que

\[ s_{n} = \sqrt{\frac{n - 1}{n}} s_{n - 1}. \]

O menor valor da amostra é \(x_{1}\), o desvio padrão viesado da amostra é
\(s_{n}\) e o desvio padrão não viesado é \(s_{n - 1}\). \\

Assim, \(\text{max}(0, 87.2 - \sqrt{187/188} \cdot 49.14)
         \leq \tilde{x} <
         87.2 \rightarrow
         38.19
         \leq \tilde{x} < 87.2\).

\subsection*{(d)} \addcontentsline{toc}{subsection}{(d)}            

\textbf{Sabendo que '\textit{skewness} é uma medida de assimetria de uma
        determinada distribuição de frequência' e que na distribuição
        normal a \textit{skewness} é zero, a estimativa da
        \textit{skewness} da distribuição dos dados apresentados no
        histograma acima deve ser positiva, negativa ou zero?
        Justifique. \\
        Note que \textit{skewness} = \(\mu_{3}/\sigma^{3}\), em que
        \(\mu_{3} = E[(X - E(X))^{3}]\) e \(\sigma\) é o desvio padrão
        populacional. Aproximadamente 60\% das observações estão abaixo da
        média.} \\

\quad
Estamos numa situação em que a mediana é menor que a média, desta forma, a
\textit{skewness} da distribuição dos dados é positiva. \\
A \textit{skewness} pode ser mensurada por

\[ \text{Skew} = 3 \cdot \frac{\bar{x} - \tilde{x}}{s_{n-1}}. \]

Como sabemos que a média é maior que a mediana, \(\bar{x} > \tilde{x}\), a
\textit{skewness} será maior que zero, Skew > 0.

\subsection*{(e)} \addcontentsline{toc}{subsection}{(e)}

\textbf{As distribuições lognormal e gama seriam duas candidatas para
        ajustar esses dados. Usando métodos computacionais e gráficos,
        como você escolheria a mais apropriada? Justifique.} \\

\quad
Pensando em ajuste de modelos de regressão, poderíamos ajustar dois
modelos lineares generalizados, um com resposta lognormal e outro com
resposta gama. O modelo com melhor ajuste, baseado numa análise de
resíduos, responderia a pergunta de qual distribuição é mais adequada aos
dados. \\

Não pensando em ajuste de modelos de regressão, dois gráficos poderiam ser
feitos pra responder essa pergunta. Um \texttt{q-q plot} que compararia as
duas funções de distribuição através da comparação gráfica dos quantis das
distribuições. Esse gráfico ajudaria na visualização de diferenças nas
caudas das distribuições. O outro gráfico útil seria um gráfico normal de
probabilidade que representaria a comparação dos quantis de cada
distribuição com uma normal padrão. Tal gráfico também seria válido na
comparação das caudas das distribuições. \\

Na Figura \ref{fig:exer1s} temos a aplicação desses dois gráficos em duas
amostras simuladas de tamanho 188 das distribuições lognormal e gama com
igual média, 90, e desvio padrão, 50. Na comparação das distribuições
observamos uma grande fuga na calda direita da distribuição lognormal. \\

Pensando no conjunto de dados do exercício e olhando para os resultados
dos gráficos, escolheríamos a distribuição lognormal como mais apropriada,
já que seus dados simulados apresentam a mesma característica dos dados
originais.

\begin{knitrout}\small
\definecolor{shadecolor}{rgb}{1, 1, 1}\color{fgcolor}\begin{figure}[H]

{\centering \includegraphics[width=\maxwidth]{iBagens/exer1s-1} 

}

\caption[q-q plot e gráficos normais de probabilidade para duas amostras simuladas de tamanho 188 de distribuições lognormal e gama de igual média, 90, e desvio padrão, 50]{q-q plot e gráficos normais de probabilidade para duas amostras simuladas de tamanho 188 de distribuições lognormal e gama de igual média, 90, e desvio padrão, 50.}\label{fig:exer1s}
\end{figure}


\end{knitrout}

\section*{Exercício 2} \addcontentsline{toc}{section}{Exercício 2}

\horrule{1pt} \\

\textbf{Seja \(X\) uma variável aleatória de distribuição gama com 
        parâmetros \(a\) e \(b\) dada por}

\begin{equation}
 \label{eq:exer2}
 f(x; a, b) = \frac{b}{\Gamma(a)} (bx)^{a - 1} e^{-bx},
\end{equation}

\textbf{tal que \(E(X) = a/b\) e \(Var(X) = a/b^{2}\). Reparametrize a
        distribuição em (\ref{eq:exer2}) de tal forma que \(E(X) = \mu\) e
        \(Var(X) = \mu/\phi\). Escreva a distribuição em (\ref{eq:exer2})
        a partir desta nova parametrização.}

\[ \mu = \frac{a}{b}
   \qquad \text{e} \qquad
   \frac{\mu}{\phi} = \frac{a}{b^{2}}, \]

logo,

\[ a = \mu b
   \qquad \text{e} \qquad
   \frac{\mu}{\phi} = \frac{\mu b}{b^{2}} = \frac{\mu}{b}
   \Rightarrow \boxed{\phi = b}
   \qquad \text{e} \qquad \boxed{a = \mu \phi}. \]

\[ \boxed{f(x; \mu, \phi) = \frac{\phi}{\Gamma(\mu \phi)}
                            (\phi x)^{\mu \phi - 1} e^{-\phi x}}. \]

\section*{Exercício 3} \addcontentsline{toc}{section}{Exercício 3}

\horrule{1pt} \\

\textbf{Suponha que indivíduos entrevistados sobre o consumo de vegetais
        informem consumos positivos ou zero (sem consumo) e que você tenha
        decidido pela distribuição gama para modelar os dados de consumo
        de vegetais positivos. Escreva uma distribuição de probabilidade
        para estes dados, considerando que \(f(x)\) é a distribuição gama
        em (\ref{eq:exer2}) com parâmetros \(\mu\) e \(\phi\), onde
        \(\mu\) é a média e \(\phi\) parâmetro de escala. Para construir a
        distribuição de probabilidade dos dados positivos e zeros use a
        idéia do modelo ZIP visto em sala de aula.}

\begin{align*}
P(X = x) &= \begin{cases}
             p & \text{ ; } x = 0 \\
             (1 - p) f(x; \mu, \phi) & \text{ ; } x \in (0, \infty)
            \end{cases}, \\
\\
\text{ZIG}(x; p, \mu, \phi) &=
 p^{\text{ I}(x = 0)}
 \left[(1-p) f(x; \mu, \phi)\right]^{\text{ I}(x > 0)}.
\end{align*}

ZIG: \textit{Zero-Inflated Gamma}. \\

Em que \(p\) é a probabilidade de ter consumo de vegetais.

\section*{Exercício 4} \addcontentsline{toc}{section}{Exercício 4}

\horrule{1pt} \\

\textbf{Considere uma normal tetravariada originada da seguinte forma:}

\begin{align}
 \label{eq:exer4}
 y_{1} &= a_{1} + e_{1} \\
 y_{2} &= a_{2} + e_{2} \nonumber
\end{align}

\textbf{em que \(E(a_{i}) = E(e_{i}) = 0\), \(Cov(a_{i}, e_{j}) = 0\),
        para \(i, j = 1, 2\). Além disso, assuma que
        \(Cov(e_{i}, e_{j}) = 0\), \(i \neq j\),
        \(Var(a_{i}) = \sigma_{a}^{2}\) e \(Var(e_{i}) = \sigma_{e}^{2}\),
        \(i, j = 1, 2\). Assuma que a distribuição de \(a_{i}\) e
        \(e_{i}\) seja normal.} 

\subsection*{(a)} \addcontentsline{toc}{subsection}{(a)}

\textbf{Qual a distribuição do vetor
        \(\textbf{d} = (y_{1}, y_{2}, a_{1}, a_{2})\)?}

\[ D = \begin{bmatrix}
        y_{1} \\
        y_{2} \\
        a_{1} \\
        a_{2}
       \end{bmatrix} = \begin{bmatrix}
                        a_{1} + e_{1} \\
                        a_{2} + e_{2} \\
                        a_{1} \\
                        a_{2}
                       \end{bmatrix} \sim N_{4} ( \mu, \Sigma), \]  

em que

\[ \mu = \begin{bmatrix}
          0 \\
          0 \\
          0 \\
          0
         \end{bmatrix}, \qquad
   \Sigma = \begin{bmatrix}
             \sigma_{a}^{2} + \sigma_{e}^{2}
             & \rho_{a_{1} a_{2}} \sigma_{a_{1}} \sigma_{a_{2}}
             & \sigma_{a}^{2}
             & \rho_{a_{1} a_{2}} \sigma_{a_{1}} \sigma_{a_{2}} \\
             \rho_{a_{1} a_{2}} \sigma_{a_{1}} \sigma_{a_{2}}
             & \sigma_{a}^{2} + \sigma_{e}^{2}
             & \rho_{a_{1} a_{2}} \sigma_{a_{1}} \sigma_{a_{2}}
             & \sigma_{a}^{2} \\
             \sigma_{a}^{2}
             & \rho_{a_{1} a_{2}} \sigma_{a_{1}} \sigma_{a_{2}}
             & \sigma_{a}^{2}
             & \rho_{a_{1} a_{2}} \sigma_{a_{1}} \sigma_{a_{2}} \\
             \rho_{a_{1} a_{2}} \sigma_{a_{1}} \sigma_{a_{2}}
             & \sigma_{a}^{2}
             & \rho_{a_{1} a_{2}} \sigma_{a_{1}} \sigma_{a_{2}}
             & \sigma_{a}^{2}
            \end{bmatrix}.
\]

Outra maneira de representar:

\[ D = \begin{bmatrix}
        y_{1} \\
        y_{2} \\
        a_{1} \\
        a_{2}
       \end{bmatrix} = \begin{bmatrix}
                        a_{1} + e_{1} \\
                        a_{2} + e_{2} \\
                        a_{1} \\
                        a_{2}
                       \end{bmatrix} = \underset{A}{\underbrace{
                                        \begin{bmatrix}
                                         1 & 0 & 1 & 0 \\
                                         0 & 1 & 0 & 1 \\
                                         1 & 0 & 0 & 0 \\
                                         0 & 1 & 0 & 0
                                        \end{bmatrix}
                                       }} \underset{X}{\underbrace{
                                           \begin{bmatrix}
                                            a_{1} \\
                                            a_{2} \\
                                            e_{1} \\
                                            e_{2}
                                           \end{bmatrix}
                                          }} = A X, \]

\[ A X \sim N_{4} ( A \mu, A \Sigma A^{'}). \]

\subsection*{(b)} \addcontentsline{toc}{subsection}{(b)}

\textbf{Usando a ideia de probabilidade condicional, mostre como encontrar
        a distribuição \(p(y_{1}, y_{2} | a_{1}, a_{2})\). Pode deixar
        indicado quando tiver alcançado uma resposta que depende das
        distribuições conhecidas no problema.}

\[ p(y_{1}, y_{2} | a_{1}, a_{2}) =
    \frac{p(y_{1}, y_{2}, a_{1}, a_{2})}{p(a_{1}, a_{2})} =
    \frac{p(y_{1}, y_{2}, a_{1}, a_{2})}{
     \int \int p(y_{1}, y_{2}, a_{1}, a_{2})\text{d} y_{1} \text{d} y_{2}}
   .
\]

\subsection*{(c)} \addcontentsline{toc}{subsection}{(c)}

\textbf{Encontre \(E(y_{i})\) e \(Var(y_{i})\) usando o método da
        iteração a partir da \(E(y_{i} | a_{i}) = a_{i}\) e
        \(Var(y_{i} | a_{i}) = \sigma_{e}^{2}\). Mostre que
        \(E(y_{i}) = 0\) e
        \(Var(y_{i}) = \sigma_{a}^{2} + \sigma_{e}^{2}\).}

\[ E(y_{i}) = E_{a_{i}}[E(y_{i} | a_{i})] = E_{a_{i}}[a_{i}] = 0. \]

\begin{align*}
 Var(y_{i}) &=
  E_{a_{i}}[Var(y_{i} | a_{i})] + Var_{a_{i}}[E(y_{i} | a_{i})] \\
            &= E_{a_{i}}[\sigma_{e}^{2}] + Var_{a_{i}}[a_{i}] \\
            &= \sigma_{e}^{2} + \sigma_{a}^{2}.
\end{align*}

\section*{Exercício 5} \addcontentsline{toc}{section}{Exercício 5}

\horrule{1pt} \\

\textbf{Considere o seguinte mecanismo de amostragem: (1) Jogue uma moeda
        em que a probabilidade de ser 'cara' é 0.3; (2) Se você observa
        cara, considere uma realização \(X \sim N(2, 1)\); se for 'coroa'
        considere uma realização \(X \sim N(8, 1)\). O código abaixo
        ilustra este experimento a partir de simulação:}

\begin{knitrout}\small
\definecolor{shadecolor}{rgb}{1, 1, 1}\color{fgcolor}\begin{kframe}
\begin{alltt}
\hlcom{# <code r> ============================================================= #}
\hlstd{coin} \hlkwb{<-}
  \hlkwd{rbinom}\hlstd{(}\hlnum{1000}
         \hlstd{,} \hlkwc{size} \hlstd{=} \hlnum{1}\hlstd{,} \hlkwc{prob} \hlstd{=} \hlnum{0.3}\hlstd{)} \hlcom{# gera 1000 amostras de uma v.a. Ber(0.3)}
\hlstd{n1} \hlkwb{<-} \hlkwd{rnorm}\hlstd{(}\hlkwd{sum}\hlstd{(coin),} \hlnum{2}\hlstd{,} \hlnum{1}\hlstd{)} \hlcom{# gera um total de soma(coin) de N(2, 1)}
\hlstd{n2} \hlkwb{<-} \hlkwd{rnorm}\hlstd{(}\hlnum{1000} \hlopt{-} \hlkwd{sum}\hlstd{(coin)}
            \hlstd{,} \hlnum{8}\hlstd{,} \hlnum{1}\hlstd{)} \hlcom{# gera um total de [1000 - soma(coin)] de N(8, 1)}
\hlcom{# </code r> ============================================================ #}
\end{alltt}
\end{kframe}
\end{knitrout}

\textbf{O histograma destas 1000 amostras é dado por}

\vspace{\fill}
\begin{figure}[H]
 \centering
  \includegraphics[width = .485\textwidth]{exer5.png}
 \caption{Histograma das 1000 v.a.'s geradas pelo processo acima.}
 \label{fig:exer5}
\end{figure}

\textbf{Considere que uma v.a. \(X\) gerada através do processo acima.
        Escreva a distribuição de probabilidade de \(X\).}

\[f_{X | \theta}(x; \theta) =
   [0.3 \cdot N(2, 1)]^{\text{I}(\theta = 0)}
   \cdot
   [0.7 \cdot N(8, 1)]^{\text{I}(\theta = 1)}, \]

Em que \(\theta\) é o resultado de jogar a moeda (cara = 0, coroa = 1). \\

\horrule{.5pt}

\vspace{\fill}

\horrule{1pt} \\

\end{document}
